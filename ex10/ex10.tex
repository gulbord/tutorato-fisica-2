\documentclass[10pt]{gulartcl}
\usepackage{../exstyle}

\title{Esercizio settimanale n. 10}
\author{Guglielmo Bordin}
\date{\today}

\begin{document}
\maketitle 

\noindent
Una spira conduttrice di area $\Sigma = \qty{15}{cm\squared}$ e resistenza
$R = \qty{6}{m\ohm}$, ferma nello spazio, è attraversata da un campo
magnetico $B$ perpendicolare alla sua superficie. L’induttanza della spira
è trascurabile. Nell’intervallo di tempo da $0$ a $\tau = \qty{2}{s}$ il
flusso del campo attraverso la spira varia come
\[
     \Phi_{B}(t) = a t (\tau - t).
 \]
\begin{itemize}
    \item Determinare le unità di misura e il valore numerico della
        costante $a$, sapendo che $B$ vale $\qty{10}{mT}$ a $t = \tau / 2$.
    \item Calcolare l’energia dissipata sulla resistenza nell’intervallo di
        tempo $(0, \tau)$.
\end{itemize} 
\end{document}
