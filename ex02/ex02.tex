\documentclass[10pt]{gulartcl}
\usepackage{../exstyle}

\title{Esercizio settimanale n. 2}
\author{Guglielmo Bordin}
\date{\today}

\begin{document}
\maketitle

\noindent
Un filo di lunghezza infinita e densità di carica uniforme $\lambda$
viene disposto nella configurazione indicata in figura. Calcolare modulo e
direzione del campo elettrico risultante nel punto $O$.

\bigbreak
\begin{center}
\begin{tikzpicture}
    \draw[dotted] (0, 3) -- (0, 2.5);
    \draw (0, 2.5) -- (0, 1) arc (180:270:1); 
    \draw (1, 0) -- (2.5, 0);
    \draw[dotted] (2.5, 0) -- (3, 0);
    \draw (1, 1) -- ++(225:1)
        node[midway, anchor=north west, inner sep=1pt] {$R$};
    \node[fill, circle, inner sep=1pt, label={$O$}] at (1, 1) {};
\end{tikzpicture}
\end{center}
\bigbreak

\begin{hint}
Per risolvere l’integrale
\[
    \mathcal{I} = \int_{0}^{\infty} \frac{R}{(x^2 + R^2)^{3/2}} \, \dd x
\]
potete usare la sostituzione $x / R = \tan u$. Vi invito a provare a fare i
conti, ma se non riuscite a proseguire passate direttamente al risultato
finale $\mathcal{I} = 1 / R$, non toglierò punti.
\end{hint}

\begin{solution}
Possiamo cominciare pensando di scomporre il campo totale nella somma
(vettoriale) dei campi prodotti dai due tratti rettilinei di filo e di
quello prodotto dal tratto curvo. Occupiamoci prima dei tratti rettilinei:
esaminiamo il problema generico del campo prodotto da un filo infinitamente
esteso lungo una sola direzione, in un punto distante $R$ dall’estremità
finita.

Facendo riferimento a figura~\ref{fig:thread-field}, partiamo considerando
un pezzetto infinitesimo di filo $\dd x$, che avrà carica $\lambda\,\dd x$
e produrrà nel punto di coordinate $x = 0$ e $y = R$ un campo di modulo
\begin{equation}
    \dd E = \frac{\lambda\, \dd x}{4 \pi \epsilon_0 (x^2 + R^2)}.
    \label{eq:thread-field-dd}
\end{equation}
Possiamo scomporre tale campo nelle componenti $x$ e $y$: seguendo il
disegno, la componente $x$ sarà negativa e pari a
\begin{equation}
    \dd E_x = - \frac{\lambda\, \dd x}{4 \pi \epsilon_0 (x^2 + R^2)}
    \cos\theta = - \frac{\lambda x}{4 \pi \epsilon_0 (x^{2} +
    R^{2})^{3/2}} \, \dd x.
\end{equation}

Da qui, possiamo integrare lungo $x$, partendo da $x = 0$ e proseguendo
fino all’infinito, per ottenere la componente $x$ del campo prodotto dal
filo intero:
\begin{equation}
    E_x = -\frac{\lambda}{4 \pi \epsilon_0} \int_{0}^{\infty}
    \frac{x}{(x^{2} + R^{2})^{3/2}}\, \dd x.
\end{equation}
Per risolvere l’integrale possiamo sfruttare la sostituzione $u^2 = x^2 +
R^2$, che differenziata restituisce
\begin{equation}
    \cancel{2}u \, \dd u = \cancel{2}x \, \dd x.
\end{equation}
Pertanto, ricordando di trasformare anche gli estremi di integrazione,
otteniamo
\begin{equation}
    E_x = - \frac{\lambda}{4 \pi \epsilon_0} \int_{R}^{\infty}
    \frac{\cancel{u}}{u^{\cancel{3}\, 2}} \, \dd u = - \frac{\lambda}{4 \pi
    \epsilon_0} \left[-\frac{1}{u}\right]_{R}^{\infty} = -
    \frac{\lambda}{4\pi\epsilon_0 R}.
\end{equation}

Ritorniamo ora all’espressione del campo infinitesimo
\eqref{eq:thread-field-dd} e consideriamone invece la componente $y$:
\begin{equation}
    \dd E_y = \frac{\lambda\, \dd x}{4\pi\epsilon_0 (x^{2} + R^{2})}
    \sin\theta,
\end{equation}
che sostituendo l’espressione per $\sin\theta$, ricavata anche qui dal
disegno, diventa
\begin{equation}
    \dd E_y = \frac{\lambda}{4\pi\epsilon_0} \frac{R}{(x^{2} +
    R^{2})^{3/2}}\, \dd x.
\end{equation}
Ora possiamo integrare lungo $x$, come prima:
\begin{equation}
    E_y = \frac{\lambda}{4\pi\epsilon_0} \int_{0}^{\infty} \frac{R}{(x^{2}
    + R^{2})^{3/2}} \, \dd x.
\end{equation}

Sfortunatamente, l’assenza di $x$ al numeratore complica notevolmente
l’integrale. Una possibile strada per risolverlo è quella indicata dal
suggerimento. Partiamo riarrangiando i termini al denominatore:
\begin{equation}
    \int_{0}^{\infty} \frac{R}{(x^{2} + R^{2})^{3/2}} \, \dd x =
    \int_{0}^{\infty} \frac{R}{[R^2(1 + x^2 / R^2)]^{3/2}}\, \dd x =
    \frac{1}{R^2} \int_{0}^{\infty} \frac{1}{(1 + x^{2} / R^{2})^{3/2}}\,
    \dd x.
\end{equation}
\end{solution}
\end{document}
