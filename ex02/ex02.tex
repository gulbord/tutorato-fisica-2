\documentclass[10pt]{gulartcl}
\usepackage{../exstyle}

\title{Esercizio settimanale n. 2}
\author{Guglielmo Bordin}
\date{\today}

\begin{document}
\maketitle

\noindent
Un filo di lunghezza infinita e densità di carica uniforme $\lambda$
viene disposto nella configurazione indicata in figura. Calcolare modulo e
direzione del campo elettrico risultante nel punto $O$.

\bigbreak
\begin{center}
\begin{tikzpicture}
    \draw[dotted] (0, 3) -- (0, 2.5);
    \draw (0, 2.5) -- (0, 1) arc (180:270:1); 
    \draw (1, 0) -- (2.5, 0);
    \draw[dotted] (2.5, 0) -- (3, 0);
    \draw (1, 1) -- ++(225:1)
        node[midway, anchor=north west, inner sep=1pt] {$R$};
    \node[fill, circle, inner sep=1pt, label={$O$}] at (1, 1) {};
\end{tikzpicture}
\end{center}
\bigbreak

\begin{hint}
Per risolvere l’integrale
\[
    \mathcal{I} = \int_{0}^{\infty} \frac{R}{(x^2 + R^2)^{3/2}} \, dx
\]
potete usare la sostituzione $x / R = \tan u$. Vi invito a provare a fare i
conti, ma se non riuscite a proseguire passate direttamente al risultato
finale $\mathcal{I} = 1 / R$, non toglierò punti.
\end{hint}

\begin{solution}
Possiamo cominciare pensando di scomporre il campo totale nella somma
(vettoriale) dei campi prodotti dai due tratti rettilinei di filo e di
quello prodotto dal tratto curvo. Occupiamoci prima dei tratti rettilinei:
esaminiamo il problema generico del campo prodotto da un filo infinitamente
esteso lungo una sola direzione, in un punto distante $R$ dall’estremità
finita.

\begin{figure}
\centering
\begin{tikzpicture}
   \draw (0, 0) -- (5, 0);
   \draw[dotted] (5, 0) -- ++(0.5, 0);
   \draw[densely dashed] (0, 0) -- ++(0, 2);
   \draw[latex-latex] (-0.2, 0) -- ++(0, 2) node[midway, anchor=east] {$R$};
   \draw[densely dashed] (3, 0) -- (0, 2);
   \draw[-latex] (0, 2) -- ++({180 - atan(2 / 3)}:1) 
       node[midway, anchor=south west] {$d\vec{E}_1$}
       coordinate[pos=1] (A);
   \draw (2.8, -0.05) -- ++(0, 0.1);
   \draw (3.2, -0.05) -- ++(0, 0.1);
   \draw[thick] (2.8, 0) -- (3.2, 0) node[midway, above] {$dx$};
   \draw[latex-latex] (0, -0.2) -- ++(3, 0) node[midway, below] {$x$};
   \draw ($(3, 0) + ({180 - atan(2 / 3)}:0.75)$) arc
       ({180 - atan(2 / 3)}:180:0.75) node[midway, anchor=east] {$\theta$};
\end{tikzpicture}
\caption{analisi del campo prodotto da uno dei tratti rettilinei di filo.}
\label{fig:thread-field}
\end{figure}

Facendo riferimento a figura~\ref{fig:thread-field}, partiamo considerando
un pezzetto infinitesimo di filo $dx$, che ha carica $\lambda\,dx$
e produce nel punto di coordinate $x = 0$ e $y = R$ un campo di modulo
\begin{equation}
    dE_1 = \frac{\lambda\, dx}{4 \pi \epsilon_0 (x^2 + R^2)}.
    \label{eq:thread-field-dd}
\end{equation}
Possiamo scomporre tale campo nelle componenti $x$ e $y$: seguendo il
disegno, la componente $x$ è negativa e pari a
\begin{equation}
    dE_{1,x} = - \frac{\lambda\, dx}{4 \pi \epsilon_0 (x^2 + R^2)}
    \cos\theta = - \frac{\lambda x}{4 \pi \epsilon_0 (x^{2} +
    R^{2})^{3/2}} \, dx.
\end{equation}

Da qui, possiamo integrare lungo $x$, partendo da $x = 0$ e proseguendo
fino all’infinito, per ottenere la componente $x$ del campo prodotto
dall’intero filo:
\begin{equation}
    E_{1,x} = -\frac{\lambda}{4 \pi \epsilon_0} \int_{0}^{\infty}
    \frac{x}{(x^{2} + R^{2})^{3/2}}\, dx.
\end{equation}
Per risolvere l’integrale possiamo sfruttare la sostituzione $u^2 = x^2 +
R^2$, che differenziata restituisce
\begin{equation}
    \cancel{2}u \, du = \cancel{2}x \, dx.
\end{equation}
Pertanto, ricordando di trasformare anche gli estremi di integrazione,
otteniamo
\begin{equation}
    E_{1,x} = - \frac{\lambda}{4 \pi \epsilon_0} \int_{R}^{\infty}
    \frac{\cancel{u}}{u^{\cancel{3}\, 2}} \, du = - \frac{\lambda}{4 \pi
    \epsilon_0} \left[-\frac{1}{u}\right]_{R}^{\infty} = -
    \frac{\lambda}{4\pi\epsilon_0 R}.
\end{equation}

Ritorniamo ora all’espressione del campo infinitesimo
\eqref{eq:thread-field-dd} e consideriamone invece la componente $y$:
\begin{equation}
    dE_{1,y} = \frac{\lambda\, dx}{4\pi\epsilon_0 (x^{2} + R^{2})}
    \sin\theta,
\end{equation}
che sostituendo l’espressione per $\sin\theta$, ricavata anche qui dal
disegno, diventa
\begin{equation}
    dE_{1,y} = \frac{\lambda}{4\pi\epsilon_0} \frac{R}{(x^{2} +
    R^{2})^{3/2}}\, dx.
\end{equation}
Ora possiamo integrare lungo $x$, come prima:
\begin{equation}
    E_{1,y} = \frac{\lambda}{4\pi\epsilon_0} \int_{0}^{\infty}
    \frac{R}{(x^{2} + R^{2})^{3/2}} \, dx.
    \label{eq:thread-field-ey}
\end{equation}

Sfortunatamente, l’assenza di $x$ al numeratore complica notevolmente
l’integrale. Una possibile strada per risolverlo è quella indicata dal
suggerimento. Partiamo riarrangiando i termini al denominatore:
\begin{equation}
    \int_{0}^{\infty} \frac{R}{(x^{2} + R^{2})^{3/2}} \, dx =
    \int_{0}^{\infty} \frac{R}{[R^2(1 + x^2 / R^2)]^{3/2}}\, dx =
    \frac{1}{R^2} \int_{0}^{\infty} \frac{1}{(1 + x^{2} / R^{2})^{3/2}}\,
    dx,
    \label{eq:ugly-integral}
\end{equation}
e usiamo la sostituzione indicata:
\begin{equation}
    x = R \tan u \implies 1 + \frac{x^{2}}{R^{2}} = 1 + \tan^2 u =
    \frac{1}{\cos^2 u}.
\end{equation}
Diferenziando $x^2 / R^2 = \tan^2 u$ otteniamo
\begin{equation}
    \frac{2 x \,dx}{R^{2}} = 2 \tan u \,\diff{}{u}(\tan u) \, du =
    \frac{2 \tan u \, du}{\cos^{2}u},
\end{equation}
che, semplificando i $2$ e dividendo per $x = R \tan u$, diventa
\begin{equation}
    \frac{dx}{R^{2}} = \frac{du}{R \cos^{2}u}.
\end{equation}
Mancano gli estremi di integrazione, che si possono trasformare così:
\begin{equation}
    x = 0 \to u = 0, \quad x = \infty \to u = \frac{\pi}{2}.
\end{equation}

Procediamo quindi a sostituire dentro \eqref{eq:ugly-integral}, ottenendo
\begin{equation}
\begin{split}
    \int_{0}^{\infty} \frac{R}{(x^{2} + R^{2})^{3/2}} \, dx 
    &= \int_{0}^{\pi/2} \cos^{3} u \, \frac{du}{R \cos^{2} u} \\
    &= \frac{1}{R} \int_{0}^{\pi/2} \cos u \, du \\
    &= \frac{1}{R} [\sin u]_{0}^{\pi/2} = \frac{1}{R}.
\end{split}
\end{equation}
Sostituiamo quindi il risultato dentro \eqref{eq:thread-field-ey}:
\begin{equation}
    E_{1, y} = \frac{\lambda}{4 \pi \epsilon_0 R}.
\end{equation}

Riassumendo, il campo prodotto dal filo in figura~\ref{fig:thread-field}
ha componenti $x$ e $y$ uguali, ed è pertanto diretto a $45$ gradi
rispetto all’orizzontale, con modulo
\begin{equation}
    E_1 = \sqrt{\left(-\frac{\lambda}{4\pi\epsilon_0
    R}\right)^{2} + \left(\frac{\lambda}{4\pi\epsilon_0 R}\right)^{2}} =
    \frac{\sqrt{2} \lambda}{4\pi\epsilon_0 R}.
\end{equation}
Il tratto di filo rettilineo posto in verticale produce un campo
$\vec{E}_2$ di uguale modulo e direzione, ma verso opposto: si faccia
riferimento a figura \ref{fig:field-of-straight-threads}. Il campo
risultante dalla somma è quindi nullo; rimane soltanto il contributo del
tratto curvo di filo, che ora calcoleremo.

\begin{figure}
\centering 
\begin{tikzpicture}
    \draw[dotted] (0, 3) -- (0, 2.5);
    \draw (0, 2.5) -- (0, 1) node[midway, anchor=east] {$2$};
    \draw (1, 0) -- (2.5, 0) node[midway, below] {$1$};
    \draw[dotted] (2.5, 0) -- (3, 0);
    \node[fill, circle, inner sep=1pt] at (1, 1) {};    
    \draw[-latex] (1, 1) -- ++(-45:0.75)
        node[midway, anchor=south west, inner sep=1pt] {$\vec{E}_2$};
    \draw[-latex] (1, 1) -- ++(135:0.75)
        node[midway, anchor=south west, inner sep=1pt] {$\vec{E}_1$};
    \draw[densely dashed] (1, 0) -- (1, 1) node[midway, anchor=east] {$R$};
    \draw[densely dashed] (0, 1) -- (1, 1);
    \draw ($(1, 1) + (135:0.4)$) arc (135:180:0.4)
        node[midway, anchor=east, inner sep=1pt] {\tiny\qty{45}{\degree}};
\end{tikzpicture}
\caption{i campi prodotti dai due tratti rettilinei di filo. In modulo sono
uguali e la direzione è la stessa ma con verso opposto, perciò la
risultante è nulla.}
\label{fig:field-of-straight-threads}
\end{figure}

\begin{figure}
\centering
\begin{tikzpicture}
    \draw[name path=arc] (0, 2) arc (180:270:2);
    \draw[name path=radius1, densely dashed] (2, 2) -- ++(205:2);
    \draw[name path=radius2, densely dashed] (2, 2) -- ++(215:2);
    \draw[-latex] (2, 2) -- ++(30:1)
        node[near end, anchor=south east, inner sep=1pt] {$d\vec{E}$};
    \draw[-latex] (2, 2) -- ++({cos(30)}, 0)
        node[midway, below] {$d\vec{E}_x$};
    \draw[-latex] (2, 2) -- ++(0, {sin(30)})
        node[midway, anchor=east] {$d\vec{E}_y$};
    \draw (2.4, 2) arc (0:30:0.4)
        node[pos=0.6, anchor=west, inner sep=1pt] {\footnotesize$\theta$};
    \draw ($(2, 2) + (205:1.3)$) arc (205:215:1.3)
        node[fill=white, midway, anchor=north east, inner sep=1pt]
        {$d\theta$};
    \fill (2, 2) circle (1pt);

    \path[name intersections={of=radius1 and arc, by=s1}];
    \path[name intersections={of=radius2 and arc, by=s2}];
    \draw (s1) -- ++(205:0.05);
    \draw (s2) -- ++(215:0.05);
    \draw[thick] (s1) arc (205:215:2)
        node[midway, anchor=north east, inner sep=1pt] {$d\ell$};
\end{tikzpicture}
\caption{impostazione grafica del calcolo del campo prodotto dal tratto
curvo di filo.}
\label{fig:curved-thread-dd}
\end{figure}

Il calcolo si può impostare come in figura~\ref{fig:curved-thread-dd}:
cominciamo considerando un tratto infinitesimo (curvo) di filo $d\ell$, che
avrà carica $\lambda \, d\ell$. Il campo prodotto in $O$ ha componenti
\begin{equation}
    dE_x = \frac{\lambda \, d\ell}{4\pi\epsilon_0 R^{2}} \cos\theta, \quad
    dE_y = \frac{\lambda \, d\ell}{4\pi\epsilon_0 R^{2}} \sin\theta.
\end{equation}
Possiamo poi riscrivere $d\ell$ come $R\, d\theta$, così da poter
integrare su $\theta$. Otteniamo perciò, per la componente $x$ totale,
\begin{equation}
    E_x = \frac{\lambda}{4 \pi \epsilon_0} \int_{0}^{\pi/2} \cos\theta \,
    d\theta = \frac{\lambda}{4 \pi \epsilon_0} [\sin\theta]_{0}^{\pi/2} =
    \frac{\lambda}{4\pi\epsilon_0 R},
\end{equation}
e per la componente $y$
\begin{equation}
    E_y = \frac{\lambda}{4 \pi \epsilon_0} \int_{0}^{\pi/2} \sin\theta \,
    d\theta = \frac{\lambda}{4\pi\epsilon_0} [-\cos\theta]_{0}^{\pi/2} =
    \frac{\lambda}{4\pi\epsilon_0 R}.
\end{equation}

Le due componenti sono uguali, pertanto l’angolo del vettore $\vec{E}$
risultante è di \qty{45}{\degree} rispetto all’asse $x$. Il modulo
vale
\begin{equation}
    E = \sqrt{\smash[b]{E_x^{2} + E_y^{2}}} = 
    \sqrt{\left(\frac{\lambda}{4\pi\epsilon_0 R}\right)^{2} +
    \left(\frac{\lambda}{4\pi\epsilon_0 R}\right)^{2}} = \frac{\sqrt{2}
    \lambda}{4 \pi\epsilon_0 R}.
\end{equation}
\end{solution}
\end{document}
