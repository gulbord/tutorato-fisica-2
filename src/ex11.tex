\documentclass[10pt]{gulartcl}
\usepackage{../exstyle}

\title{Esercizio settimanale n. 11}
\author{Guglielmo Bordin}
\date{\today}

\begin{document}
\maketitle 

\noindent
Una regione di spazio vuoto è pervasa da un campo elettrico variabile nel
tempo descritto dall’espressione
\[
    \vec{E} = E_0 \sin\left[\frac{2\pi}{\lambda}(z - ct)\right]
          (\uvec{x} + \uvec{y}).
\]
Derivare l’espressione del campo magnetico usando le leggi di Maxwell,
sapendo che il suo modulo vale $B_0 = E_0 / c$ nell’origine a $t = 0$.

\bigbreak
\begin{hint}
È sufficiente usare solo una delle quattro equazioni in forma differenziale
(scegliete la più appropriata), e ricordarsi della costante di
integrazione.
\end{hint}
\end{document}
