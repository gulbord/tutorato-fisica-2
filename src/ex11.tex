\documentclass[10pt]{gulartcl}
\usepackage{../exstyle}

\title{Esercizio settimanale n. 11}
\author{Guglielmo Bordin}
\date{\today}

\begin{document}
\maketitle 

\noindent
Una regione di spazio vuoto è pervasa da un campo elettrico variabile nel
tempo descritto dall’espressione
\[
    \vec{E} = E_0 \sin\left[\frac{2\pi}{\lambda}(z - ct)\right]
          (\uvec{x} + \uvec{y}).
\]
Derivare l’espressione del campo magnetico usando le leggi di Maxwell,
sapendo che il suo modulo vale $B_0 = E_0 / c$ nell’origine a $t = 0$.

\bigbreak
\begin{hint}
È sufficiente usare solo una delle quattro equazioni in forma differenziale
(scegliete la più appropriata), e ricordarsi della costante di
integrazione.
\end{hint}

\begin{solution}
\textbf{Il testo che vi ho consegnato aveva un problema:} la
condizione al contorno «$B_0$ nell’origina a $t = 0$» non è sufficiente a
garantire una soluzione univoca. Una consegna corretta (vedere più avanti
perché) sarebbe stata:
\begin{quote}
    […] Derivare l’espressione del campo magnetico usando le leggi di
    Maxwell, \emph{tralasciando termini costanti nel tempo}.
\end{quote}
Ne terrò conto nella correzione, mi scuso per il
disagio.

Conviene partire dalla legge di Faraday in forma differenziale:
\begin{equation}
    \curl{E} = -\pdiff{\vec{B}}{t},
\end{equation}
integrando rispetto al tempo entrambi i lati dell’equazione otteniamo
\begin{equation}
    \vec{B}(t) = -\int \curl{E}(t) \, dt + \vec{B}_0.
    \label{eq:B(t)-final}
\end{equation}
Possiamo chiamare $\vec{B}_0$ la costante di integrazione, che potrà
dipendere da $x$, $y$, $z$ ma non dal tempo, la variabile su cui stiamo
integrando.

Dobbiamo quindi calcolare il rotore del campo elettrico. Per farlo
possiamo usare il trucco del determinante:
\begin{equation}
\begin{split} 
    \curl{E} &= \det
    \begin{pmatrix}
        \uvec{x} & \uvec{y} & \uvec{z} \\
        \partial_x & \partial_y & \partial_z \\
        E_x & E_y & E_z
    \end{pmatrix} \\[0.25\baselineskip]
             &= \left(\pdiff{E_z}{y} - \pdiff{E_y}{z}\right) \uvec{x}
              + \left(\pdiff{E_x}{z} - \pdiff{E_z}{x}\right) \uvec{y}
              + \left(\pdiff{E_y}{x} - \pdiff{E_x}{y}\right) \uvec{z}.
\end{split}
\end{equation}
Nel nostro caso la componente $z$ è nulla, mentre $E_x$ ed $E_y$ dipendono
soltanto da $z$. Ciò che rimane è quindi
\begin{equation}
    \curl{E} = -\pdiff{E_y}{z} \uvec{x} + \pdiff{E_x}{z} \uvec{y}
             = -\frac{2\pi}{\lambda} E_0
               \cos\left[\frac{2\pi}{\lambda}(z - ct)\right]
               (\uvec{x} - \uvec{y}).
\end{equation}

Riprendiamo ora l’equazione \eqref{eq:B(t)-final} e inseriamo quanto
trovato:
\begin{equation}
    \vec{B}(t)
    = \vec{B}_0 +\frac{2\pi}{\lambda} E_0 (\uvec{x} - \uvec{y})
      \int \cos\left[\frac{2\pi}{\lambda} (z - ct)\right] \, dt.
\end{equation}
Qui possiamo effettuare la sostituzione di variabili
\begin{equation}
    \frac{2\pi}{\lambda} (z - ct) = u,
\end{equation}
che applicata ai differenziali dà
\begin{equation}
    -\frac{2 \pi c}{\lambda} \, dt = du \implies dt
    = -\frac{\lambda}{2 \pi c} \, du.
\end{equation}
Sostituendo nell’integrale otteniamo
\begin{equation}
\begin{split}
    \vec{B}(t)
    &= \vec{B}_0 + \frac{E_0}{c} (\uvec{y} - \uvec{x})
       \int \cos(u) \, du \\
    &= \vec{B}_0 + \frac{E_0}{c} \sin(u) (\uvec{y} - \uvec{x}) \\
    &= \vec{B}_0 + \frac{E_0}{c} \sin\left[\frac{2\pi}{\lambda}(z - ct)\right]
       (\uvec{y} - \uvec{x}).
\end{split}
\end{equation}
Qui nel secondo passaggio abbiamo inglobato la nuova costante di
integrazione in $\vec{B}_0$, e risostituito la definizione di $u$ nel
terzo.

Ora dobbiamo ricordare che $\vec{B}_0$, per quanto sappiamo, è costante
solo rispetto al tempo: possiamo scrivere per completezza $\vec{B}_0(x, y,
z)$. Nell’origine a $t = 0$ il termine con il seno è nullo e rimane
soltanto $\vec{B}_0(0, 0, 0)$ ($\vec{B}_0$ calcolato nell’origine).
Con le informazioni fornite possiamo concludere che $\vec{B}_0(0, 0, 0)$ è
un vettore di modulo $B_0 = E_0 / c$, ma niente di più! Seguiamo quindi
la consegna «corretta» e tralasciamo i campi costanti nel tempo, ossia
poniamo $\vec{B}_0 = \vec{0}$. La soluzione è dunque
\begin{equation}
    \vec{B}(t) = \frac{E_0}{c} \sin\left[\frac{2\pi}{\lambda}(z -
    ct)\right] (\uvec{y} - \uvec{x}).
\end{equation}
\end{solution}
\end{document}
