\documentclass[10pt]{gulartcl}
\usepackage{../exstyle}

\title{Esercizio settimanale n. 4}
\author{Guglielmo Bordin}
\date{\today}

\begin{document}
\maketitle

\noindent
Su un filo sottile di lunghezza infinita è presente una densità di carica
uniforme $\lambda = \qty{-2}{nC \per cm}$. All’esterno del cavo, coassiale
a esso, è posto un guscio cilindrico di lunghezza infinita e carico
uniformemente, di raggio interno $a = \qty{1}{cm}$ e raggio esterno $b =
\qty{3}{cm}$.  Calcolare la densità di carica volumetrica $\rho$ del guscio
cilindrico sapendo che il campo elettrico totale al suo esterno è nullo.

\bigbreak
\begin{center}
\begin{tikzpicture}[>=latex]
    % external cylinder
    \draw[name path=bot-outer-ell] (-1.2, 0) arc (180:360:1.2 and 0.4); 
    \draw[densely dashed] (1.2, 0) arc (0:180:1.2 and 0.4);
    \draw (-1.2, 0) -- +(0, 3) (1.2, 0) -- +(0, 3);
    \draw[pattern=north east lines] (0, 3) ellipse (1.2 and 0.4);
    
    % inner cylinder
    \draw[densely dashed] (0, 0) ellipse (0.8 and 0.2);
    \draw[densely dashed] (-0.8, 0) -- +(0, 3) (0.8, 0) -- +(0, 3);
    \draw[name path=top-inner-ell, fill=white] (0, 3) ellipse (0.8 and 0.2);

    % thread
    \path[name path=thread] (0, -1) -- (0, 3);
    \draw[densely dashed,
          name intersections={of=thread and bot-outer-ell, by=z1},
          name intersections={of=thread and top-inner-ell, by=z2}]
        (z1) -- (z2);
    \draw (0, -1) -- (z1) (z2) -- +(0, 1) node[pos=0.9, right] {$\lambda$};

    \node[right] at (1.2, 2) {$\rho$};
    \draw[<->] (0, 1.8) -- +(-0.8, 0) node[midway, below] {$a$}; 
    \draw[<->] (0, 1) -- +(1.2, 0) node[midway, below] {$b$};
\end{tikzpicture} 
\end{center}

\begin{solution}
Per rispondere conviene applicare il teorema di Gauss. Consideriamo una
superficie cilindrica $\Sigma$ di raggio $r$ maggiore del raggio esterno
$b$ del guscio cilindrico e altezza $h$ arbitraria. Il flusso del campo
elettrico attraverso $\Sigma$ è, per ipotesi, nullo: il flusso parziale
attraverso la superficie superiore e quella inferiore è nullo per via della
direzione delle linee di campo, che è radiale per ragioni di simmetria, e
anche il flusso attraverso la superficie laterale è nullo perché vogliamo
$\vec{E} = \vec{0}$ all’esterno del cilindro.

Per la legge di Gauss abbiamo
\begin{equation}
    \Phi_{\Sigma}(\vec{E}) = \frac{Q_{\mathrm{int}}}{\epsilon_0}, 
\end{equation}
dove $Q_{\mathrm{int}}$ è la carica totale racchiusa dalla superficie
costruita. Per avere $\Phi_{E} = 0$ dobbiamo quindi calcolare
$Q_{\mathrm{int}}$ e azzerarla.

La carica dovuta al filo è semplicemente la densità lineare $\lambda$
moltiplicata per la lunghezza del tratto «tagliato» da $\Sigma$. La carica
dovuta al guscio cilindrico è data invece dalla densità volumetrica $\rho$
moltiplicata per il volume tagliato, cioè $\pi h (b^{2} -
a^{2})$.\footnote{Per ricavare questa formula si può pensare al volume di
    un cilindro di raggio $b$ e altezza $h$, che è $\pi b^{2} h$, e
    sottrarci il volume di un cilindro di uguale altezza ma raggio
    $a$, cioè $\pi a^{2} h$.} Arriviamo dunque a
\begin{equation}
    \lambda \cancel{h} + \rho \pi \cancel{h} (b^{2} - a^{2}) = 0,
\end{equation}
e quindi la densità cercata è data da
\begin{equation}
    \rho = - \frac{\lambda}{\pi(b^{2} - a^{2})} = \frac{\qty{2}{nC \per
    cm}}{\pi [(\qty{3}{cm})^{2} - (\qty{1}{cm})^{2}]} = \qty{80}{pC \per
    \cm\cubed} = \qty{80}{\micro C \per m\cubed}.
\end{equation}
\end{solution}
\end{document}
