\documentclass[10pt]{gulartcl}
\usepackage{../exstyle}

\title{Esercizio settimanale n. 5}
\author{Guglielmo Bordin}
\date{\today}

\begin{document}
\maketitle

\noindent
Una sfera metallica conduttrice ha una densità di carica superficiale
$\sigma = \qty{25}{nC\per m\squared}$ e un raggio $R$ minore di $2$ metri.
A distanza $a = \qty{2}{m}$ dal centro della sfera il potenziale elettrico
vale \qty{500}{V} e l’intensità del campo elettrico \qty{250}{V \per m} (il
riferimento per il potenziale è all’infinito, dove lo si considera pari a
$0$).

\begin{itemize}
    \item Qual è il raggio della sfera?
    \item Qual è il suo potenziale?
\end{itemize}

\begin{solution}
Il raggio si può trovare a partire dal valore del campo o del potenziale a
\qty{2}{m} dal centro della sfera. Si possono fare i conti sia con l’uno
che con l’altro; scegliamo il potenziale.

Sappiamo che il potenziale prodotto da una sfera carica al suo esterno ha
simmetria radiale e vale, in un punto a distanza $r$ dal suo centro,
\begin{equation}
    V(r) = \frac{Q}{4\pi\epsilon_0 r},
\end{equation}
dove $Q$ è la carica totale della sfera, che riscriviamo come $4\pi R^2 \sigma$
dato che disponiamo della densità superficiale. Noi sappiamo il valore in
$r = a$, quindi sostituiamo:
\begin{equation}
    V(a) = \qty{500}{V} = \frac{Q}{4\pi\epsilon_0 a} =
    \frac{\sigma\cancel{4 \pi} R^2}{\cancel{4\pi} \epsilon_0 a},
\end{equation}
e riarrangiando i termini otteniamo il valore del raggio $R$:
\begin{equation}
    R = \sqrt{\frac{\epsilon_0 a V(a)}{\sigma}} = \qty{0.6}{m}.
\end{equation}

La sfera è conduttrice, dunque ogni suo punto si trova allo stesso
potenziale, che chia\-mia\-mo $V_{\mathrm{s}}$. Per calcolarlo possiamo usare
l’espressione del potenziale in un punto sulla superficie (il valore sarà
lo stesso anche lì):
\begin{equation}
    V_{\mathrm{s}} = \frac{Q}{4\pi\epsilon_0 R} = \frac{\sigma \cancel{4\pi}
    R^{\bcancel{2}}}{\cancel{4\pi}\epsilon_0 \bcancel{R}} = \frac{\sigma
    R}{\epsilon_0} = \qty{1700}{V}.
\end{equation}
\end{solution}
\end{document}

