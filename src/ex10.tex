\documentclass[10pt]{gulartcl}
\usepackage{../exstyle}

\title{Esercizio settimanale n. 10}
\author{Guglielmo Bordin}
\date{\today}

\begin{document}
\maketitle 

\noindent
Una spira conduttrice di area $\Sigma = \qty{15}{cm^2}$ e resistenza
$R = \qty{6}{m\ohm}$, ferma nello spazio, è attraversata da un campo
magnetico $B$ perpendicolare alla sua superficie. L’induttanza della spira
è trascurabile. Nell’intervallo di tempo da $0$ a $\tau = \qty{2}{s}$ il
flusso del campo attraverso la spira varia come
\[
     \Phi_{B}(t) = a t (\tau - t).
 \]
\begin{itemize}
    \item Determinare le unità di misura e il valore numerico della
        costante $a$, sapendo che $B$ vale $\qty{10}{mT}$ a $t = \tau / 2$.
    \item Calcolare l’energia dissipata sulla resistenza nell’intervallo di
        tempo $(0, \tau)$.
\end{itemize} 

\begin{solution}
Il flusso magnetico ha unità $\unit{Wb} = \unit{T.m^2}$, dunque $a$
deve avere unità \unit{Wb \per s^2} per bilanciare il tempo al
quadrato:
\begin{equation}
    [a] = \frac{[\Phi_B]}{[t][\tau - t]} = \unit{Wb \per s^2}
        = \unit{V \per s}.
\end{equation}

Poiché la superficie $\Sigma$ è perpendicolare alla direzione del campo
magnetico, possiamo scrivere il flusso come
\begin{equation}
    \Phi_B(t) = B(t) \Sigma,
\end{equation}
e dunque, utilizzando l’espressione fornita,
\begin{equation}
    a = \frac{\Phi_B(t)}{t(\tau - t)} = \frac{B(t) \Sigma}{t(\tau - t)}.
\end{equation}
Sostituendo $t = \tau / 2$ e $B = \qty{10}{mT}$ otteniamo
\begin{equation}
    a = \frac{B(\tau / 2) \Sigma}{\tau^2 / 4}
      = \frac{4(\qty{1e-2}{T}) (\qty{1.5e-3}{m^2})}{\qty{4}{s^2}}
      = \qty{1.5e-5}{Wb \per s^2}.
\end{equation}

Il flusso di $\vec{B}$ attraverso la spira varia nel tempo, dunque vi si
genera una forza elettromotrice $\femi$ la cui espressione
è data dalla legge di Faraday:
\begin{equation}
    \femi(t) = - \diff{}{t} \Phi_B(t).
\end{equation}
Il segno meno di Lenz in questo caso è irrilevante e possiamo ometterlo dai
conti successivi; faremo calcoli energetici, quindi le direzioni e i segni
non ci interessano. $\femi$ corrisponde alla differenza di potenziale che
viene a formarsi ai capi della resistenza: la potenza dissipata per effetto
Joule sarà quindi
\begin{equation}
    P(t) = \frac{\femi^2(t)}{R}.
\end{equation}

Calcoliamo dunque innanzitutto la f.e.m. indotta, derivando il flusso
rispetto al tempo:
\begin{equation}
    \femi(t) = \diff{}{t} [a t (\tau - t)] = a \diff{}{t} (\tau t - t^2)
             = a (\tau - 2t).
\end{equation}
Poi, osserviamo che la \emph{potenza} è l’energia (in questo caso,
l’energia dissipata) \emph{per unità di tempo}. Per avere quindi l’energia
dissipata complessivamente in un certo intervallo di tempo, dobbiamo
integrare. In questo caso non basta moltiplicare per la durata
dell’intervallo, perché la f.e.m. non è costante nel tempo.

Procediamo quindi con l’integrazione di $P(t)$ da $t = 0$ a $t = \tau$, per
ottenere l’energia dissipata $W$:
\begin{equation}
    W = \frac{a^2}{R} \int_{0}^{\tau} (\tau - 2t)^2 \, dt.
\end{equation}
Qui possiamo fare la seguente sostituzione:
\begin{equation}
    2t - \tau = u, \quad dt = du / 2,
\end{equation}
che implica (ricordandosi di trasformare anche gli estremi di integrazione)
\begin{equation}
    W = \frac{a^2}{2R} \int_{-\tau}^{+\tau} u^2 \, du
      = \frac{a^2}{2R} \biggl[\frac{u^3}{3}\biggr]_{u = -\tau}^{u = +\tau}
      = \frac{a^2}{2R}\biggl(\frac{\tau^3}{3} + \frac{\tau^3}{3}\biggr)
      = \frac{a^2 \tau^3}{3 R}.
\end{equation}
Sostituendo i valori numerici otteniamo
\begin{equation}
    W = \frac{(\qty{1.5e-5}{V \per s})^2 (\qty{2}{s})^3}%
             {3 (\qty{6e-3}{\ohm})}
      = \qty{1e-7}{J}.
\end{equation}
\end{solution}
\end{document}
