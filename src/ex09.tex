\documentclass[10pt]{gulartcl}
\usepackage{../exstyle}

\title{Esercizio settimanale n. 9}
\author{Guglielmo Bordin}
\date{\today}

\begin{document}
\maketitle 

\noindent
Il nucleo metallico della Terra si estende dal centro fino a circa metà del
suo raggio $R = \qty{6400}{km}$. Al polo nord magnetico si osserva un campo
magnetico di circa $\qty{0.5e-4}{T}$, diretto perpendicolarmente al
terreno. Immaginiamo che il meccanismo di generazione del campo sia
descrivibile tramite una corrente elettrica che fluisce in un anello
attorno al nucleo, all’altezza del suo equatore. Quanto dovrebbe valere
tale corrente per spiegare il dato del campo al polo?

\begin{solution}
Se possiamo considerare la corrente localizzata interamente sull’equatore
del nucleo, allora possiamo usare l’espressione per il campo magnetico
lungo l’asse di una spira circolare percorsa da corrente:
\begin{equation}
    B(r) = \frac{\mu_0 i a^2}{2(r^2 + a^2)^{3 / 2}},
\end{equation}
dove $a$ è il raggio della spira e $r$ la distanza dal suo centro lungo
l’asse.

Guardando ai dati del problema, dobbiamo porre $a \approx R / 2$, il
raggio del nucleo, e $r = R$, la distanza dal polo al centro della Terra.
Otteniamo così
\begin{equation}
    B = \frac{\mu_0 i (R / 2)^2}{2[R^2 + (R / 2)^2]^{3 / 2}}
      = \frac{\mu_0 i}{5 \sqrt{5} R}.
\end{equation}
Noi sappiamo $B$ e cerchiamo $i$, quindi invertiamo:
\begin{equation}
    i = \frac{5 \sqrt{5} R B}{\mu_0}
      = \frac{5 \sqrt{5} \, (\qty{6.4e6}{m})(\qty{5e-5}{T})}%
             {\qty{4\pi e-7}{T.m\per A}}
      = \qty{2.8e9}{A}.
\end{equation}
\end{solution}
\end{document}
