\documentclass[10pt]{gulartcl}
\usepackage[version=4]{mhchem}
\usepackage{../exstyle}

\title{Esercizio settimanale n. 7}
\author{Guglielmo Bordin}
\date{\today}

\begin{document}
\maketitle 

\noindent
La resistività dell’acqua di mare è di circa \qty{0.25}{\ohm m}. I
portatori di carica sono principalmente ioni \ce{Na+} e \ce{Cl-}, e di
ciascuno di essi ce ne sono circa \num{3e26} per \unit{m\cubed}. Se
riempiamo un tubo di plastica lungo \qty{2}{m} con acqua di mare e
colleghiamo una batteria da \qty{12}{V} a degli elettrodi presenti sulle
due estremità, quale sarà la velocità di deriva degli ioni risultante?

\begin{solution}
Sappiamo che la densità di corrente equivale a
\begin{equation}
    j = n e v_\mathrm{d},
    \label{eq:j=nev}
\end{equation}
dove $v_{\mathrm{d}}$ è la velocità di deriva che cerchiamo e $n$ è la
densità (numerica) di portatori di carica. Nel nostro caso i portatori di
carica sono due, gli ioni sodio e gli ioni cloruro, ciascuno con
concentrazione \qty{3e26}{\per\meter\cubed}: dunque $n =
\qty{6e26}{\per\meter\cubed}$.

Cerchiamo dunque di ottenere un’altra espressione per $j$ in base ai dati
forniti. Conosciamo la lunghezza del tubo e la differenza di potenziale tra
le due estremità, e dunque anche il campo elettrico. Possiamo perciò usare
la legge di Ohm in versione microscopica,
\begin{equation}
    j = \sigma E = \frac{1}{\rho} \frac{V}{L},
\end{equation}
dove $L = \qty{2}{m}$ è la lunghezza del tubo.

Alternativamente, possiamo vedere $j$ come l’intensità di corrente divisa
per l’area del tubo, usare la legge di Ohm macroscopica e l’espressione per
la resistenza di un tubo:
\begin{equation}
    j = \frac{i}{A} = \frac{V / R}{A}
      = \frac{V}{\bcancel{A}} \frac{\bcancel{A}}{\rho L}
      = \frac{V}{\rho L}.
\end{equation}

Inserendo il risultato nell’equazione \eqref{eq:j=nev} otteniamo
\begin{equation}
\begin{split}
    v_{\mathrm{d}} &= \frac{j}{n e} = \frac{V}{n e \rho L} \\
    &= \frac{\qty{12}{V}}{(\qty{6e26}{\per\meter\cubed}) (\qty{1.6e-19}{C})
    (\qty{0.25}{\ohm m}) (\qty{2}{m})} = \qty{2.5e-7}{m\per s}.
\end{split}
\end{equation}
\end{solution}
\end{document}
