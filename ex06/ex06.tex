\documentclass[10pt]{gulartcl}
\usepackage{../exstyle}

\title{Esercizio settimanale n. 6}
\author{Guglielmo Bordin}
\date{\today}

\begin{document}
\maketitle

\noindent
Tre armature piane conduttrici, di area $\Sigma = \qty{0.25}{m\squared}$,
sono disposte rispettivamente nei piani di equazione $x_{1} =
\qty{1.5}{cm}$, $x_{2} = \qty{3.0}{cm}$, $x_{3} = \qty{4.5}{cm}$.
L’armatura centrale è collegata a terra, mentre le armature 1 e 3 sono
isolate, con densità di carica $\sigma_{1} = \qty{5}{\micro C \per
m\squared}$ e $\sigma_{3} = \qty{-2}{\micro C \per m\squared}$
rispettivamente. Tra le armature 1 e 2 e tra le armature 2 e 3 si trovano
due dielettrici di costanti dielettriche relative $\kappa_{1} = 3$ e
$\kappa_{2} = 2$ rispettivamente.
\begin{itemize}
    \item Determinare il potenziale $V(x)$ in funzione della posizione $x$
        all’interno del sistema di armature, in particolare il valore di
        $V(x_{3})$.
\end{itemize}
Le armature 1 e 3 vengono poi collegate tra loro da un filo conduttore.
\begin{itemize}
    \item Determinare le densità di carica $\sigma'_{1}$ e $\sigma'_{3}$
        presenti ora sulle armature 1 e 3. 
\end{itemize}
\begin{hint}
Quando sono collegate le armature si trovano allo stesso potenziale.
\end{hint}

\bigbreak

% plates + dielectrics code
\newcommand\platewidth{0.1}
\newcommand\capacitor[1]{
    \begin{scope}[xshift=#1cm]
        % plates
        \draw (-\platewidth, 0) rectangle (0, 3);
        \draw (1.5 - \platewidth / 2, 0)
              rectangle
              (1.5 + \platewidth / 2, 3);
        \draw (3, 0) rectangle (3 + \platewidth, 3);

        % dielectrics
        \fill[fill=gray!70] (0, 0) rectangle (1.5 - \platewidth / 2, 3);
        \fill[fill=gray!30] (1.5 + \platewidth / 2, 0) rectangle (3, 3);

        % ground
        \draw (1.5, 0) -- ++(0, -0.8)
            ++(-0.4, 0) -- ++(0.8, 0) 
            ++(-0.65, -0.2) -- ++(0.5, 0)
            ++(-0.35, -0.2) -- ++(0.2, 0);

        % labels
        \node at (0.75 - \platewidth / 4, 2) {$\kappa_{1}$}; 
        \node at (2.25 + \platewidth / 4, 2) {$\kappa_{2}$};
    \end{scope}
}

\begin{center}
\begin{tikzpicture}[scale=0.8]
    % first configuration
    \capacitor{0};
    \node[left] at (-\platewidth, 1) {$\sigma_{1}$};
    \node[right] at (3 + \platewidth, 1) {$\sigma_{3}$};

    % second configuration
    \capacitor{6};
    \begin{scope}[xshift=6cm]
        \draw (-\platewidth, 2.5) -- ++(-0.5, 0) -- ++(0, 1)
            -- ++(4 + 2 * \platewidth, 0) -- ++(0, -1) -- ++(-0.5, 0);
        \node[left] at (-\platewidth, 1) {$\sigma'_{1}$};
        \node[right] at (3 + \platewidth, 1) {$\sigma'_{3}$};
    \end{scope}
\end{tikzpicture}
\end{center}

\begin{solution}
L’armatura centrale, seppur non carica come quelle esterne, è anch’essa
conduttrice: dunque per induzione elettrostatica sulla sua superficie
affacciata al dielettrico 1 si deposita una densità di carica $-\sigma_1$
(negativa), mentre sulla superficie affacciata all’altro dielettrico si
deposita una densità di carica $-\sigma_3$ (positiva). In sostanza, il
sistema diventa equivalente a due condensatori in serie; nella regione
occupata dal dielettrico 1 il campo è quello di un condensatore con carica
$\pm \sigma_1$ sulle armature, cioè $\sigma_1 / \kappa_1 \epsilon_0$, e
analogamente per la regione occupata dal dielettrico 2, dove il campo è
quello di un condensatore con carica $\pm \abs{\sigma_3}$ sulle armature,
cioè $\abs{\sigma_3} / \kappa_2 \epsilon_0$.

Per calcolare il potenziale integriamo il campo elettrico. Innanzitutto,
poniamo come riferimento di potenziale nullo l’armatura centrale collegata
a terra. Poi possiamo iniziare dalla regione 1, dove il campo è $\sigma_1 /
\kappa_1 \epsilon_0$ diretto verso destra:
\begin{alignat}{2}
    V(x) &= \cancel{V(x_2)}
            -\int_{x_2}^{x} \frac{\sigma_1}{\kappa_1 \epsilon_0} \, dx'
          = \frac{\sigma_1}{\kappa_1 \epsilon_0} (x_2 - x)
         && \quad \text{per } x \in [x_1, x_2]. \\
\intertext{Proseguiamo nella regione 2, dove il campo è $\abs{\sigma_3} /
\kappa_2 \epsilon_0 = -\sigma_3 / \kappa_2 \epsilon_0$ diretto verso
destra:}
    V(x) &= \cancel{V(x_2)} -\int_{x_2}^{x}
            \biggl(-\frac{\sigma_3}{\kappa_2 \epsilon_0}\biggr) \, dx'
          = \frac{\sigma_3}{\kappa_2 \epsilon_0}(x - x_2)
         && \quad \text{per } x \in [x_2, x_3].
\end{alignat}
Nelle considerazioni sui segni è importante tenere a mente che se il
potenziale centrale è nullo, sull’armatura carica positivamente ci
aspettiamo un potenziale \emph{positivo} e su quella carica negativamente
un potenziale \emph{negativo} (come appunto succede per le espressioni
scritte sopra).

Il potenziale in $x = x_3$ vale dunque
\begin{equation}
    V(x_3) = \frac{\sigma_3}{\kappa_2 \epsilon_0}(x_3 - x_2)
          = - \frac{\qty{2e-6}{C \per m\squared}}{2(\qty{8.9e-12}{F/m})} 
            (\qty{1.5e-2}{m})
          = -\qty{1.7}{kV}.
\end{equation}

Quando le armature 1 e 3 vengono collegate dal filo conduttore, la carica
presente su di esse si redistribuisce affinché raggiungano lo stesso
potenziale. Abbiamo quindi due condizioni da imporre per trovare
$\sigma'_1$ e $\sigma'_3$: la conservazione della carica \emph{totale}:
\begin{equation}
    \sigma'_1 + \sigma'_3 = \sigma_1 + \sigma_3,
    \label{eq:system-1st}
\end{equation}
e il raggiungimento dello stesso potenziale, ossia $V'(x_1) = V'(x_3)$
(ricicliamo quindi le espressioni ricavate in precedenza sostituendo le
densità di carica):
\begin{equation}
    \frac{\sigma'_1}{\kappa_1 \epsilon_0} (x_2 - x_1) =
    \frac{\sigma'_3}{\kappa_2 \epsilon_0} (x_3 - x_2).
    \label{eq:system-2nd}
\end{equation}

Poiché $x_3 - x_2 = x_2 - x_1$ troviamo subito dalla \eqref{eq:system-2nd}
\begin{equation}
    \sigma'_3 = \frac{\kappa_2}{\kappa_1} \sigma'_1,
\end{equation}
che sostituita nella \eqref{eq:system-1st} dà
\begin{equation}
    \biggl(1 + \frac{\kappa_2}{\kappa_1}\biggr) \sigma'_1 = \sigma_1 +
    \sigma_3.
\end{equation}
Dunque in definitiva:
\begin{align}
    \sigma'_1 &= \frac{\kappa_1}{\kappa_1 + \kappa_2}(\sigma_1 + \sigma_3) =
    \frac{3}{5} (\qty{3}{\micro C \per m\squared}) = \qty{1.8}{\micro C \per
    m\squared}, \\
    \sigma'_3 &= \frac{\kappa_2}{\kappa_1 + \kappa_2}(\sigma_1 + \sigma_3)
    = \frac{2}{5} (\qty{3}{\micro C\per m\squared}) = \qty{1.2}{\micro
    C\per m\squared}.
\end{align}
\end{solution}
\end{document}
