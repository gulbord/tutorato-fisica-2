\documentclass[10pt]{gulartcl}
\usepackage{../exstyle}

\title{Esercizio settimanale n. 6}
\author{Guglielmo Bordin}
\date{\today}

\begin{document}
\maketitle

\noindent
Tre armature piane conduttrici, di area $\Sigma = \qty{0.25}{m\squared}$,
sono disposte rispettivamente nei piani di equazione $x_{1} =
\qty{1.5}{cm}$, $x_{2} = \qty{3.0}{cm}$, $x_{3} = \qty{4.5}{cm}$.
L’armatura centrale è collegata a terra, mentre le armature 1 e 3 sono
isolate, con densità di carica $\sigma_{1} = \qty{5}{\micro C \per
m\squared}$ e $\sigma_{3} = \qty{-2}{\micro C \per m\squared}$
rispettivamente. Tra le armature 1 e 2 e tra le armature 2 e 3 si trovano
due dielettrici di costanti dielettriche relative $\kappa_{1} = 3$ e
$\kappa_{2} = 2$ rispettivamente.
\begin{itemize}
    \item Determinare il potenziale $V(x)$ in funzione della posizione $x$
        all’interno del sistema di armature, in particolare il valore di
        $V(x_{3})$.
\end{itemize}
Le armature 1 e 3 vengono poi collegate tra loro da un filo conduttore.
\begin{itemize}
    \item Determinare le densità di carica $\sigma'_{1}$ e $\sigma'_{3}$
        presenti ora sulle armature 1 e 3. 
\end{itemize}
\begin{hint}
Quando sono collegate le armature si trovano allo stesso potenziale.
\end{hint}

\bigbreak

% plates + dielectrics code
\newcommand\platewidth{0.1}
\newcommand\capacitor[1]{
    \begin{scope}[xshift=#1cm]
        % plates
        \draw (-\platewidth, 0) rectangle (0, 3);
        \draw (1.5 - \platewidth / 2, 0)
              rectangle
              (1.5 + \platewidth / 2, 3);
        \draw (3, 0) rectangle (3 + \platewidth, 3);

        % dielectrics
        \fill[fill=gray!70] (0, 0) rectangle (1.5 - \platewidth / 2, 3);
        \fill[fill=gray!30] (1.5 + \platewidth / 2, 0) rectangle (3, 3);

        % ground
        \draw (1.5, 0) -- ++(0, -0.8)
            ++(-0.4, 0) -- ++(0.8, 0) 
            ++(-0.65, -0.2) -- ++(0.5, 0)
            ++(-0.35, -0.2) -- ++(0.2, 0);

        % labels
        \node at (0.75 - \platewidth / 4, 2) {$\kappa_{1}$}; 
        \node at (2.25 + \platewidth / 4, 2) {$\kappa_{2}$};
    \end{scope}
}

\begin{center}
\begin{tikzpicture}[scale=0.8]
    % first configuration
    \capacitor{0};
    \node[left] at (-\platewidth, 1) {$\sigma_{1}$};
    \node[right] at (3 + \platewidth, 1) {$\sigma_{3}$};

    % second configuration
    \capacitor{6};
    \begin{scope}[xshift=6cm]
        \draw (-\platewidth, 2.5) -- ++(-0.5, 0) -- ++(0, 1)
            -- ++(4 + 2 * \platewidth, 0) -- ++(0, -1) -- ++(-0.5, 0);
        \node[left] at (-\platewidth, 1) {$\sigma'_{1}$};
        \node[right] at (3 + \platewidth, 1) {$\sigma'_{3}$};
    \end{scope}
\end{tikzpicture}
\end{center}
\end{document}
