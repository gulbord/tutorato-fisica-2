\documentclass[10pt]{gulartcl}
\usepackage{../exstyle}

\title{Esercizio settimanale n. 8}
\author{Guglielmo Bordin}
\date{\today}

\begin{document}
\maketitle 

\noindent
Un rudimentale cannone protonico è dotato di un sistema di accelerazione e
puntamento schematizzato in figura. Un fascio di protoni attraversa un
canale dove è accelerato da una differenza di potenziale $V =
\qty{30}{kV}$. All’uscita del canale i protoni entrano in una regione
circolare di raggio $R = \qty{1}{m}$ permeata da un campo magnetico $B$
uniforme e ortogonale al piano in cui si muovono i protoni, di intensità e
verso regolabili.

Sotto l’azione del campo magnetico i protoni percorrono
un tratto di circonferenza di raggio $r$ all’interno della regione
magnetizzata, per poi uscire dal cerchio in una direzione che forma un
angolo $\theta$ con la direzione iniziale.
\begin{itemize}
    \item Fissato il potenziale $V$, che valore di $B$ bisogna impostare
        per sparare i protoni a un certo angolo $\theta$? In particolare,
        calcolare il valore necessario per sparare a \qty{45}{\degree}
        verso sinistra.
    \item Perché cambiamo $B$ e non $V$ per decidere la direzione del
        fascio? A cosa può servire variare $V$ invece?
\end{itemize}
\begin{hint}
A un certo punto servirà scrivere il raggio di curvatura $r$ in funzione di
$R$ e $\theta$: osservate il disegno per trovare la relazione geometrica.
\end{hint}

\bigbreak
\begin{center}
\begin{tikzpicture}[>=latex, scale=0.9]
    \small
    % lengths
    \newdimen\radius
    \radius=2cm
    \newdimen\channel
    \channel=2.5cm
    % coordinates
    \coordinate (O) at (0, 0);
    \coordinate (A) at ($(O) - (\radius, 0)$);
    \coordinate (B) at ($(O) + ({\radius * cos(60)}, {-\radius * sin(60)})$);

    % entrance channel
    \draw (O) circle (2);
    \draw[pattern=north east lines]
        ($(A) + (0, 0.3)$) rectangle ++(-\channel, 0.2);
    \draw[pattern=north east lines]
        ($(A) - (0, 0.3)$) rectangle ++(-\channel, -0.2);

    % particle path with arrow
    \begin{scope}[
        decoration={
            markings,
            mark=at position 0.5 with {\arrow{latex}}
        }
    ]
        \draw[postaction={decorate}] ($(A) - (\channel, 0)$) -- (A);
        \draw[postaction={decorate}] (A) arc (90:30:{\radius * sqrt(3)});

        % exit channel
        \begin{scope}[shift={(B)}, rotate=-60]
            \draw[postaction={decorate}]
                (0, 0) -- (\channel, 0);
            \draw[pattern=north east lines]
                (0, 0.3) rectangle (\channel, 0.5);
            \draw[pattern=north east lines]
                (0, -0.3) rectangle (\channel, -0.5);
            \draw[->] (2 * \channel / 3, 0.7)
                arc (0:15:{\radius + 2 * \channel / 3});
            \draw[->] (2 * \channel / 3, -0.7)
                arc (0:-15:{\radius + 2 * \channel / 3});
        \end{scope}
    \end{scope}

    % name paths to get center of curvature
    \path[name path=vertical] (A) -- ++(0, -4);
    \path[name path=diagonal] (B) -- ++(210:4);

    % help lines
    \draw[densely dashed,
          name intersections={of=vertical and diagonal, by=C}]
        (A) -- node[midway, above] {$R$} (O)
            -- node[midway, right] {$R$} (B)
            -- node[midway, anchor=north west] {$r$} (C)
            -- node[midway, left] {$r$} (A)
        (C) -- (O) -- ++(\radius, 0);

    % angles
    \draw ($(O) + (0.5, 0)$) arc (0:-60:0.5) node[midway, right] {$\theta$};
    \draw ($(C) + (0, 0.5)$) arc (90:30:0.5)
        node[pos=0.4, anchor=south west, circle,
             inner sep=1.5pt, fill=white] {$\theta$};
    \draw ($(A) + (0.2, 0)$) -- ++(0, -0.2) -- ++(-0.2, 0);
    \draw (B) ++(210:0.2) -- ++(120:0.2) -- ++(30:0.2);

    % other labels
    \fill ($(O) + (0, \radius / 2)$) circle (1.5pt)
        node[right] {$\vec{B}$};
    \node[left] at ($(A) - (\channel, 0)$) {$\mathrm{p}^{+}$};
    \draw[<->] ($(A) + (-\channel, 0.8)$)
        -- node[midway, fill=white] {$V$} ++(\channel, 0);
\end{tikzpicture}
\end{center}

\begin{solution}
I protoni vengono inizialmente accelerati dalla differenza di potenziale
$V$ fino a raggiungere una certa velocità $v$. Quando entrano nella regione
con campo magnetico iniziano a risentire della forza di Lorentz, che se
$\vec{B}$ ha verso uscente (come nel disegno) li fa curvare verso destra
lungo un arco di circonferenza di raggio $r$, senza variare il modulo della
velocità.

La forza di Lorentz agisce come forza centripeta lungo la traiettoria
circolare. Considerando un singolo protone, possiamo dunque scrivere
\begin{equation}
    e B v = \frac{m v^2}{r},
    \label{eq:lorentz}
\end{equation}
dove $r$ è il raggio di curvatura, e $m$ è pari a \qty{1.67e-27}{kg}. Per
determinare la velocità usiamo la conservazione dell’energia: l’energia
cinetica all’entrata del cerchio è pari all’energia potenziale
elettrostatica dovuta alla differenza di potenziale $V$,
\begin{equation}
    \frac{1}{2} m v^2 = e V.
\end{equation}
Da qui troviamo subito
\begin{equation}
    v = \sqrt{\frac{2 e V}{m}}.
    \label{eq:v-expr}
\end{equation}

Usando queste relazioni, riprendiamo l’equazione \eqref{eq:lorentz}
ricordando che stiamo cercando un’espressione di $B$ in funzione degli
altri parametri del sistema. Scriviamo quindi
\begin{equation}
    B = \frac{m v}{e r} = \frac{m}{e r} \sqrt{\frac{2 e V}{m}}
      = \frac{1}{r} \sqrt{\frac{2 m V}{e}}.
    \label{eq:B-expr}
\end{equation}

Rimane da riscrivere $r$ in funzione di altre grandezze che possano essere
direttamente misurate. Come suggerito, queste sono il raggio del cerchio
$R$ e l’angolo di tiro $\theta$. Se osserviamo il disegno, notiamo che i
due triangoli rettangoli dai bordi tratteggiati sono uguali (hanno un lato
in comune e i restanti uguali); dunque la linea tratteggiata che parte dal
centro del cerchio di raggio $R$ taglia in due l’angolo $\theta$ in basso.
Concentrandoci su uno dei due triangoli rettangoli, possiamo quindi
scrivere
\begin{equation}
    \tan\frac{\theta}{2}
        = \frac{\text{cateto opposto}}{\text{cateto adiacente}} = \frac{R}{r}.
\end{equation}
Sostituendo nell’equazione \eqref{eq:B-expr} troviamo
\begin{equation}
    B(\theta) = \frac{1}{R} \sqrt{\frac{2 m V}{e}} \tan \frac{\theta}{2}. 
    \label{eq:B(theta)}
\end{equation}

Se vogliamo il valore corrispondente a un angolo di \qty{45}{\degree}
\emph{verso sinistra} dobbiamo so\-sti\-tu\-i\-re $\theta = -\pi / 4$, perché il
disegno presuppone un’orientazione positiva in senso orario per $\theta$.
Dunque
\begin{equation}
    B(-\pi / 4) = \frac{1}{\qty{1}{m}} \sqrt{\frac{2 (\qty{1.67e-27}{kg})
    (\qty{30e3}{V})}{\qty{1.60e-19}{C}}} (-\num{0.414}) = \qty{-0.010}{T}.
\end{equation}

Dalla teoria, o anche invertendo l’equazione \eqref{eq:B(theta)} ponendo
$\theta$ in funzione del resto, possiamo concludere che anche variando $V$
potremmo variare l’angolo di tiro. La differenza principale sta nel fatto
che, fissato un valore di $B$, saremmo in grado di sparare soltanto verso
destra o verso sinistra! Ad esempio, con un vettore $\vec{B}$ uscente dal foglio i
protoni non possono che curvare verso destra. Variando il potenziale
possiamo \emph{cambiare la velocità}, e di conseguenza anche l’angolo di
uscita, ma non possiamo «sconfinare» oltre la linea corrispondente a
$\theta = 0$. Se cambiassimo il segno di $V$ i protoni neanche
arriverebbero al cerchio, perché verrebbero accelerati nella direzione
opposta. In definitiva, conviene agire su $V$ per regolare la velocità di
uscita dei protoni e su $B$ per regolarne la direzione. 
\end{solution}
\end{document}
