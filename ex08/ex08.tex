\documentclass[10pt]{gulartcl}
\usepackage{../exstyle}

\title{Esercizio settimanale n. 8}
\author{Guglielmo Bordin}
\date{\today}

\begin{document}
\maketitle 

\noindent
Un rudimentale cannone protonico è dotato di un sistema di accelerazione e
puntamento schematizzato in figura. Un fascio di protoni attraversa un
canale dove è accelerato da una differenza di potenziale $V =
\qty{30}{kV}$. All’uscita del canale i protoni entrano in una regione
circolare di raggio $R = \qty{1}{m}$ permeata da un campo magnetico $B$
uniforme e ortogonale al piano in cui si muovono i protoni, di intensità e
verso regolabili.

Sotto l’azione del campo magnetico i protoni percorrono
un tratto di circonferenza di raggio $r$ all’interno della regione
magnetizzata, per poi uscire dal cerchio in una direzione che forma un
angolo $\theta$ con la direzione iniziale.
\begin{itemize}
    \item Fissato il potenziale $V$, che valore di $B$ bisogna impostare
        per sparare i protoni a un certo angolo $\theta$? In particolare,
        calcolare il valore necessario per sparare a \qty{45}{\degree}
        verso sinistra.
    \item Perché cambiamo $B$ e non $V$ per decidere la direzione del
        fascio? A cosa può servire variare $V$ invece?
\end{itemize}
\begin{hint}
A un certo punto servirà scrivere il raggio di curvatura $r$ in funzione di
$R$ e $\theta$: osservate il disegno per trovare la relazione geometrica.
\end{hint}

\bigbreak
\begin{center}
\begin{tikzpicture}[>=latex, scale=0.9]
    \small
    % lengths
    \newdimen\radius
    \radius=2cm
    \newdimen\channel
    \channel=2.5cm
    % coordinates
    \coordinate (O) at (0, 0);
    \coordinate (A) at ($(O) - (\radius, 0)$);
    \coordinate (B) at ($(O) + ({\radius * cos(60)}, {-\radius * sin(60)})$);

    % entrance channel
    \draw (O) circle (2);
    \draw[pattern=north east lines]
        ($(A) + (0, 0.3)$) rectangle ++(-\channel, 0.2);
    \draw[pattern=north east lines]
        ($(A) - (0, 0.3)$) rectangle ++(-\channel, -0.2);

    % particle path with arrow
    \begin{scope}[
        decoration={
            markings,
            mark=at position 0.5 with {\arrow{latex}}
        }
    ]
        \draw[postaction={decorate}] ($(A) - (\channel, 0)$) -- (A);
        \draw[postaction={decorate}] (A) arc (90:30:{\radius * sqrt(3)});

        % exit channel
        \begin{scope}[shift={(B)}, rotate=-60]
            \draw[postaction={decorate}]
                (0, 0) -- (\channel, 0);
            \draw[pattern=north east lines]
                (0, 0.3) rectangle (\channel, 0.5);
            \draw[pattern=north east lines]
                (0, -0.3) rectangle (\channel, -0.5);
            \draw[->] (2 * \channel / 3, 0.7)
                arc (0:15:{\radius + 2 * \channel / 3});
            \draw[->] (2 * \channel / 3, -0.7)
                arc (0:-15:{\radius + 2 * \channel / 3});
        \end{scope}
    \end{scope}

    % name paths to get center of curvature
    \path[name path=vertical] (A) -- ++(0, -10);
    \path[name path=diagonal] (B) -- ++(210:10);

    % help lines
    \draw[densely dashed,
          name intersections={of=vertical and diagonal, by=C}]
        (A) -- node[midway, above] {$R$} (O)
            -- node[midway, right] {$R$} (B)
            -- node[midway, anchor=north west] {$r$} (C)
            -- node[midway, left] {$r$} (A)
        (C) -- (O) -- ++(\radius, 0);

    % angles
    \draw ($(O) + (0.5, 0)$) arc (0:-60:0.5) node[midway, right] {$\theta$};
    \draw ($(C) + (0, 0.5)$) arc (90:30:0.5)
        node[pos=0.4, anchor=south west, circle,
             inner sep=1.5pt, fill=white] {$\theta$};
    \draw ($(A) + (0.2, 0)$) -- ++(0, -0.2) -- ++(-0.2, 0);
    \draw (B) ++(210:0.2) -- ++(120:0.2) -- ++(30:0.2);

    % other labels
    \fill ($(O) + (0, \radius / 2)$) circle (1.5pt)
        node[right] {$\vec{B}$};
    \node[left] at ($(A) - (\channel, 0)$) {$\mathrm{p}^{+}$};
    \draw[<->] ($(A) + (-\channel, 0.8)$)
        -- node[midway, fill=white] {$V$} ++(\channel, 0);
\end{tikzpicture}
\end{center}
\end{document}
